%%%%%%%%%%%%%%%%%%%%%%%%%%%%%%%%%%%%%%%%%
% Engineering Calculation Paper
% LaTeX Template
% Version 1.0 (20/1/13)
%
% This template has been downloaded from:
% http://www.LaTeXTemplates.com
%
% Original author:
% Dmitry Volynkin (dim_voly@yahoo.com.au)
%
% License:
% CC BY-NC-SA 3.0 (http://creativecommons.org/licenses/by-nc-sa/3.0/)
%
%%%%%%%%%%%%%%%%%%%%%%%%%%%%%%%%%%%%%%%%%

%----------------------------------------------------------------------------------------
%	PACKAGES AND OTHER DOCUMENT CONFIGURATIONS
%----------------------------------------------------------------------------------------

\documentclass[10pt,a4paper]{article} % Use A4 paper with a 12pt font size - different paper sizes will require manual recalculation of page margins and border positions

\usepackage{marginnote} % Required for margin notes
\usepackage{wallpaper} % Required to set each page to have a background
\usepackage{lastpage} % Required to print the total number of pages
\usepackage[left=1.3cm,right=4.6cm,top=1.8cm,bottom=4.0cm,marginparwidth=3.4cm]{geometry} % Adjust page margins
\usepackage{amsmath} % Required for equation customization
\usepackage{amssymb} % Required to include mathematical symbols
\usepackage{xcolor} % Required to specify colors by name
\usepackage{listings}

\usepackage{fancyhdr} % Required to customize headers
\setlength{\headheight}{80pt} % Increase the size of the header to accommodate meta-information
\pagestyle{fancy}\fancyhf{} % Use the custom header specified below
\renewcommand{\headrulewidth}{0pt} % Remove the default horizontal rule under the header

\setlength{\parindent}{0cm} % Remove paragraph indentation
\newcommand{\tab}{\hspace*{2em}} % Defines a new command for some horizontal space

\newcommand\BackgroundStructure{ % Command to specify the background of each page
\setlength{\unitlength}{1mm} % Set the unit length to millimeters

\definecolor{amaranth}{rgb}{0.9, 0.17, 0.31}
\definecolor{babyblueeyes}{rgb}{0.63, 0.79, 0.95}
\definecolor{beige}{rgb}{0.96, 0.96, 0.86}
\definecolor{bittersweet}{rgb}{1.0, 0.44, 0.37}
\definecolor{black}{rgb}{0.0, 0.0, 0.0}
\definecolor{bleudefrance}{rgb}{0.19, 0.55, 0.91}
\definecolor{bostonuniversityred}{rgb}{0.8, 0.0, 0.0}
\definecolor{brightube}{rgb}{0.82, 0.62, 0.91}
\definecolor{darkseagreen}{rgb}{0.56, 0.74, 0.56}
\definecolor{lavender}{rgb}{0.9, 0.9, 0.98}
\definecolor{mayablue}{rgb}{0.45, 0.76, 0.98}
\definecolor{cadmiumgreen}{rgb}{0.0, 0.42, 0.24}
\definecolor{almond}{rgb}{0.94, 0.87, 0.8}
\definecolor{antiquewhite}{rgb}{0.98, 0.92, 0.84}
\definecolor{ashgrey}{rgb}{0.7, 0.75, 0.71}
\definecolor{babyblueeyes}{rgb}{0.63, 0.79, 0.95}
\definecolor{beige}{rgb}{0.96, 0.96, 0.86}
\definecolor{blond}{rgb}{0.98, 0.94, 0.75}
\definecolor{cream}{rgb}{1.0, 0.99, 0.82}
\definecolor{eggshell}{rgb}{0.94, 0.92, 0.84}

\setlength\fboxsep{0mm} % Adjusts the distance between the frameboxes and the borderlines
\setlength\fboxrule{0.5mm} % Increase the thickness of the border line
\put(10, 10){\fcolorbox{black}{darkseagreen!10}{\framebox(155,247){}}} % Main content box
\put(165, 10){\fcolorbox{black}{blond!10}{\framebox(37,247){}}} % Margin box
\put(10, 262){\fcolorbox{black}{amaranth!10}{\framebox(192, 25){}}} % Header box
\put(170, 263){\includegraphics[height=23mm,keepaspectratio]{csm}} % Logo box - maximum height/width: 
}

%----------------------------------------------------------------------------------------
%	HEADER INFORMATION
%----------------------------------------------------------------------------------------

\fancyhead[L]{
\begin{tabular}{l l | l l} % The header is a table with 4 columns
\textbf{CSC447:} & Parallel Programming &  \textbf{Name:} & Samer Saber  \\ % Project name and page count
\textbf{Lab 1:}  & Pi using Pthreads &   \textbf{ID:}  & 201401460\\ % Project name and page count
\textbf{Date:}&   \today &  \textbf{Page:} & \thepage/\pageref{LastPage}  \\ % Project name and page count
\textbf{Spring 2022} & & & \\ % Version and reviewed date
\end{tabular}}


 

%&  & \textbf{Date} & 27/11/2012 \\ % Job number and last updated date


%----------------------------------------------------------------------------------------

\begin{document}

\AddToShipoutPicture{\BackgroundStructure} % Set the background of each page to that specified above in the header information section

%----------------------------------------------------------------------------------------
%	DOCUMENT CONTENT
%----------------------------------------------------------------------------------------

\subsection*{Abstract}

Making observations on each code and extracting the running time.
\section{Introduction}

Implementing all the codes and observing the results after each run to check the performance.

\section{Implementation}

1- Reproducing the serial code:

\begin{lstlisting}[language=C, caption=C Example]
#include <stdio.h>
#include <stdlib.h>
#include <time.h>

int main()
{
	int 		array_size;
	int 		counter;
	int * 		rand_arr;
	double		duration;
	clock_t		start;
	clock_t		end;

	array_size	= 10000000;
	counter		= 0;
	rand_arr	= calloc(array_size, sizeof(int));
	srand(time(NULL));
	start		= clock();

	for (int i = 0; i < array_size; i++)
    {
        rand_arr[i] = rand() % 10;
        if (rand_arr[i] == 3)
        {
            counter++;
        }
    }

	end		= clock();
	duration= ((double)(end - start) / CLOCKS_PER_SEC) * 1000;
	printf("There are %d 3s and it takes %fms", counter, duration);
	return 0;
}
\end{lstlisting}

2- Implementing data race:

\begin{lstlisting}[language=C, caption=C Example]
#include <stdio.h>
#include <stdlib.h>
#include <time.h>
#include <pthread.h>

#define MaxThreads 1000
void* count3s_thread(void* id);
pthread_t tid[MaxThreads];

int t;         /* number of threads */
int * array;
int length;
int count;

void count3s()
{
   int i;
   count = 0;
   /* Create t threads */
   for(i = 0; i < t; i++)
   {
      pthread_create(&tid[i], NULL, count3s_thread, (void*)i);
   }

   /*** wait for the threads to finish ***/
   for(i = 0; i < t; i++)
   {
      pthread_join(tid[i], NULL);
   }
}

void* count3s_thread(void* id)
{
   int i;
   /* Compute portion of the array that this thread should work on */
   int length_per_thread = length / t;
   int start = (intptr_t)id * length_per_thread;

   for(i = start; i < start+length_per_thread; i++)
   {
      if(array[i] == 3)
      {
         count++;
      }
   }
   return 0;
}


int main(int argc, char *argv[])
{
   int i;
   length = 1048576;  /*  2^20  */
   t = 40;  /*** be sure that t divides length!! ***/

   array = calloc(length, sizeof(int));

   /* initialize the array with random integers between 0 and 9 */
   srand(time(NULL));  /* seed the random number generator with current time */
   for(i = 0; i < length; i++)
   {
      array[i] = rand()%10;
   }

   clock_t start = clock();
   count3s();
   clock_t end = clock();
   double time_spent = ((double)(end - start) / CLOCKS_PER_SEC) * 1000.0;
   printf("It takes %fms\n", time_spent);

   printf("Parallel: The number of 3's is %d\n", count);

   count = 0;
   for (i = 0; i < length; i++)
      if (array[i] == 3)
         count++;
   printf("Serial: The number of 3's is %d\n", count);

   return 0;
}
\end{lstlisting}

3- Implementing data race with locks only:

\begin{lstlisting}[language=C, caption=C Example]
#include <stdio.h>
#include <stdlib.h>
#include <time.h>
#include <pthread.h>

#define MaxThreads 1000
void* count3s_thread(void* id);
pthread_t tid[MaxThreads];

int t;         /* number of threads */
int * array;
int length;
int count;

pthread_mutex_t m = PTHREAD_MUTEX_INITIALIZER;

void count3s()
{
   int i;
   count = 0;
   /* Create t threads */
   for(i = 0; i < t; i++)
   {
      pthread_create(&tid[i], NULL, count3s_thread, (void*)i);
   }

   /*** wait for the threads to finish ***/
   for(i = 0; i < t; i++)
   {
      pthread_join(tid[i], NULL);
   }
}

void* count3s_thread(void* id)
{
   int i;
   /* Compute portion of the array that this thread should work on */
   int length_per_thread = length / t;
   int start = (intptr_t)id * length_per_thread;

   for(i = start; i < start+length_per_thread; i++)
   {
      if(array[i] == 3)
      {
         pthread_mutex_lock(&m);
         count++;
         pthread_mutex_unlock(&m);
      }
   }
   return 0;
}


int main(int argc, char *argv[])
{
   int i;
   length = 1048576;  /*  2^20  */
   t = 40;  /*** be sure that t divides length!! ***/

   array = calloc(length, sizeof(int));

   /* initialize the array with random integers between 0 and 9 */
   srand(time(NULL));  /* seed the random number generator with current time */
   for(i = 0; i < length; i++)
   {
      array[i] = rand()%10;
   }

   clock_t start = clock();
   count3s();
   clock_t end = clock();
   double time_spent = ((double)(end - start) / CLOCKS_PER_SEC) * 1000.0;
   printf("It takes %fms\n", time_spent);

   printf("Parallel: The number of 3's is %d\n", count);

   count = 0;
   for (i = 0; i < length; i++)
      if (array[i] == 3)
         count++;
   printf("Serial: The number of 3's is %d\n", count);

   return 0;
}
\end{lstlisting}

4- data race with locks and padding:

\begin{lstlisting}[language=C, caption=C Example]
#include <stdio.h>
#include <stdlib.h>
#include <time.h>
#include <pthread.h>

#define MaxThreads 1000
void* count3s_thread(void* id);
pthread_t tid[MaxThreads];

int t;         /* number of threads */
int * array;
int length;
int count;

struct padded_int
{
    int value;
    char padding[60];
} private_count[MaxThreads];
pthread_mutex_t m = PTHREAD_MUTEX_INITIALIZER;

void count3s()
{
   int i;
   count = 0;
   /* Create t threads */
   for(i = 0; i < t; i++)
   {
      pthread_create(&tid[i], NULL, count3s_thread, (void*)i);
   }

   /*** wait for the threads to finish ***/
   for(i = 0; i < t; i++)
   {
      pthread_join(tid[i], NULL);
   }
}

void* count3s_thread(void* id)
{
   int i;
   /* Compute portion of the array that this thread should work on */
   int length_per_thread = length / t;
   int start = (int)id * length_per_thread;

   for(i = start; i < start+length_per_thread; i++)
   {
      if(array[i] == 3)
      {
         private_count[(int)id].value++;
      }
   }
   pthread_mutex_lock(&m);
   count += private_count[(int)id].value;
   pthread_mutex_unlock(&m);

   return 0;
}


int main(int argc, char *argv[])
{
   int i;
   length = 1048576;  /*  2^20  */
   t = 40;  /*** be sure that t divides length!! ***/

   array = calloc(length, sizeof(int));

   /* initialize the array with random integers between 0 and 9 */
   srand(time(NULL));  /* seed the random number generator with current time */
   for(i = 0; i < length; i++)
   {
      array[i] = rand()%10;
   }

   clock_t start = clock();
   count3s();
   clock_t end = clock();
   double time_spent = ((double)(end - start) / CLOCKS_PER_SEC) * 1000.0;
   printf("It takes %fms\n", time_spent);

   printf("Parallel: The number of 3's is %d\n", count);

   count = 0;
   for (i = 0; i < length; i++)
      if (array[i] == 3)
         count++;
   printf("Serial: The number of 3's is %d\n", count);

   return 0;
}
\end{lstlisting}

\section{Experimental Platform}
Windows 10, Sublime text editor and a GCC compiler

\section{Results}

no padding:

\begin{figure}[htp]
    \centering
    \includegraphics[width=12cm]{no-padding.png}
    \caption{A screenshot of the terminal for no padding}
    \label{fig:termianl}
\end{figure}

no padding but locks only:

\begin{figure}[htp]
    \centering
    \includegraphics[width=12cm]{no-padding-locks.png}
    \caption{A screenshot of the terminal for no padding but locks only}
    \label{fig:termianl}
\end{figure}

With padding and locks:

\begin{figure}[htp]
    \centering
    \includegraphics[width=12cm]{with-padding-lock.png}
    \caption{A screenshot of the terminal for padding and locks only}
    \label{fig:termianl}
\end{figure}

\section{Discussion}

With the increase of the number of iterations, the value
of pi becomes more accurate, and the run-time of the program increases. Thus, the more random numbers we generate to increase the accuracy of pi, the more the duration will
increase (bad performance)


\section{Conclusion}
The more random numbers we generate to increase the accuracy of pi, the more the duration will
increase (bad performance)

%----------------------------------------------------------------------------------------

\end{document}