%%%%%%%%%%%%%%%%%%%%%%%%%%%%%%%%%%%%%%%%%
% Engineering Calculation Paper
% LaTeX Template
% Version 1.0 (20/1/13)
%
% This template has been downloaded from:
% http://www.LaTeXTemplates.com
%
% Original author:
% Dmitry Volynkin (dim_voly@yahoo.com.au)
%
% License:
% CC BY-NC-SA 3.0 (http://creativecommons.org/licenses/by-nc-sa/3.0/)
%
%%%%%%%%%%%%%%%%%%%%%%%%%%%%%%%%%%%%%%%%%

%----------------------------------------------------------------------------------------
%	PACKAGES AND OTHER DOCUMENT CONFIGURATIONS
%----------------------------------------------------------------------------------------

\documentclass[10pt,a4paper]{article} % Use A4 paper with a 12pt font size - different paper sizes will require manual recalculation of page margins and border positions

\usepackage{marginnote} % Required for margin notes
\usepackage{wallpaper} % Required to set each page to have a background
\usepackage{lastpage} % Required to print the total number of pages
\usepackage[left=1.3cm,right=4.6cm,top=1.8cm,bottom=4.0cm,marginparwidth=3.4cm]{geometry} % Adjust page margins
\usepackage{amsmath} % Required for equation customization
\usepackage{amssymb} % Required to include mathematical symbols
\usepackage{xcolor} % Required to specify colors by name
\usepackage{listings}

\usepackage{fancyhdr} % Required to customize headers
\setlength{\headheight}{80pt} % Increase the size of the header to accommodate meta-information
\pagestyle{fancy}\fancyhf{} % Use the custom header specified below
\renewcommand{\headrulewidth}{0pt} % Remove the default horizontal rule under the header

\setlength{\parindent}{0cm} % Remove paragraph indentation
\newcommand{\tab}{\hspace*{2em}} % Defines a new command for some horizontal space

\newcommand\BackgroundStructure{ % Command to specify the background of each page
\setlength{\unitlength}{1mm} % Set the unit length to millimeters

\definecolor{amaranth}{rgb}{0.9, 0.17, 0.31}
\definecolor{babyblueeyes}{rgb}{0.63, 0.79, 0.95}
\definecolor{beige}{rgb}{0.96, 0.96, 0.86}
\definecolor{bittersweet}{rgb}{1.0, 0.44, 0.37}
\definecolor{black}{rgb}{0.0, 0.0, 0.0}
\definecolor{bleudefrance}{rgb}{0.19, 0.55, 0.91}
\definecolor{bostonuniversityred}{rgb}{0.8, 0.0, 0.0}
\definecolor{brightube}{rgb}{0.82, 0.62, 0.91}
\definecolor{darkseagreen}{rgb}{0.56, 0.74, 0.56}
\definecolor{lavender}{rgb}{0.9, 0.9, 0.98}
\definecolor{mayablue}{rgb}{0.45, 0.76, 0.98}
\definecolor{cadmiumgreen}{rgb}{0.0, 0.42, 0.24}
\definecolor{almond}{rgb}{0.94, 0.87, 0.8}
\definecolor{antiquewhite}{rgb}{0.98, 0.92, 0.84}
\definecolor{ashgrey}{rgb}{0.7, 0.75, 0.71}
\definecolor{babyblueeyes}{rgb}{0.63, 0.79, 0.95}
\definecolor{beige}{rgb}{0.96, 0.96, 0.86}
\definecolor{blond}{rgb}{0.98, 0.94, 0.75}
\definecolor{cream}{rgb}{1.0, 0.99, 0.82}
\definecolor{eggshell}{rgb}{0.94, 0.92, 0.84}

\setlength\fboxsep{0mm} % Adjusts the distance between the frameboxes and the borderlines
\setlength\fboxrule{0.5mm} % Increase the thickness of the border line
\put(10, 10){\fcolorbox{black}{darkseagreen!10}{\framebox(155,247){}}} % Main content box
\put(165, 10){\fcolorbox{black}{blond!10}{\framebox(37,247){}}} % Margin box
\put(10, 262){\fcolorbox{black}{amaranth!10}{\framebox(192, 25){}}} % Header box
\put(170, 263){\includegraphics[height=23mm,keepaspectratio]{csm}} % Logo box - maximum height/width: 
}

%----------------------------------------------------------------------------------------
%	HEADER INFORMATION
%----------------------------------------------------------------------------------------

\fancyhead[L]{
\begin{tabular}{l l | l l} % The header is a table with 4 columns
\textbf{CSC447:} & Parallel Programming &  \textbf{Name:} & Samer Saber  \\ % Project name and page count
\textbf{Lab 2:}  & Pi using Pthreads &   \textbf{ID:}  & 201401460\\ % Project name and page count
\textbf{Date:}&   \today &  \textbf{Page:} & \thepage/\pageref{LastPage}  \\ % Project name and page count
\textbf{Spring 2022} & & & \\ % Version and reviewed date
\end{tabular}}


 

%&  & \textbf{Date} & 27/11/2012 \\ % Job number and last updated date


%----------------------------------------------------------------------------------------

\begin{document}

\AddToShipoutPicture{\BackgroundStructure} % Set the background of each page to that specified above in the header information section

%----------------------------------------------------------------------------------------
%	DOCUMENT CONTENT
%----------------------------------------------------------------------------------------

\subsection*{Abstract}

Reproducing the serial code by creating random arrays of different sizes and observing the duration.
\section{Introduction}

Reproducing the serial code by creating random arrays of sizes 1000, 10000, 100000, and 10000000. Then observing
the value of Pi on each array size, and checking how long it took.

\section{Implementation}

At each iteration we're generating a random number from 0 to 1.

\begin{lstlisting}[language=C, caption=C Example]
#include <stdio.h>
#include <stdlib.h>
#include <time.h>
#include <math.h>

int main()
{
	int		i;//number of iterations
	double  duration;
	double  pi;
	clock_t start;
	clock_t end;
	int 	counter;

	i 		= 10000000;
	start	= clock();
	counter = 0;

	srand(time(NULL));
	
	for (int index = 0; index < i; index++)
    {
        double a = (rand() % 543) / (double)(542);
        double b = (rand() % 543) / (double)(542);
        if (sqrt((a * a) + (b * b)) <= 1)
        {
            counter++;
        }
    }

	end		= clock();
	duration= ((double)(end - start) / CLOCKS_PER_SEC) * 1000;
	pi 		= 4.0 * ((double)counter / i);
	printf("Serial of pi is %f and it takes %fms\n", pi, duration);

	return 0;
}
\end{lstlisting}

\section{Experimental Platform}
Windows 10, Sublime text editor and a GCC compiler

\section{Results}
\begin{tabular}{ |p{2cm}||p{2cm}|p{2cm}|  }
 \hline
 \multicolumn{3}{|c|}{Serial Performance} \\
 \hline
 Size& Pi&Time (ms)\\
 \hline
 1000   & 3,112000    &0.0000\\
 10000&   3.132800  & 0.0000\\
 100000& 3.133080 & 93.0000\\
 100000000    &3.137810 & 2656.0000\\
 \hline
\end{tabular}

\section{Discussion}

With the increase of the number of iterations, the value
of pi becomes more accurate, and the run-time of the program increases. Thus, the more random numbers we generate to increase the accuracy of pi, the more the duration will
increase (bad performance)


\section{Conclusion}
The more random numbers we generate to increase the accuracy of pi, the more the duration will
increase (bad performance)

%----------------------------------------------------------------------------------------

\end{document}