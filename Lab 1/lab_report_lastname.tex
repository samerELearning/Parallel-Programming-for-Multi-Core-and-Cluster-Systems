%%%%%%%%%%%%%%%%%%%%%%%%%%%%%%%%%%%%%%%%%
% Engineering Calculation Paper
% LaTeX Template
% Version 1.0 (20/1/13)
%
% This template has been downloaded from:
% http://www.LaTeXTemplates.com
%
% Original author:
% Dmitry Volynkin (dim_voly@yahoo.com.au)
%
% License:
% CC BY-NC-SA 3.0 (http://creativecommons.org/licenses/by-nc-sa/3.0/)
%
%%%%%%%%%%%%%%%%%%%%%%%%%%%%%%%%%%%%%%%%%

%----------------------------------------------------------------------------------------
%	PACKAGES AND OTHER DOCUMENT CONFIGURATIONS
%----------------------------------------------------------------------------------------

\documentclass[10pt,a4paper]{article} % Use A4 paper with a 12pt font size - different paper sizes will require manual recalculation of page margins and border positions

\usepackage{marginnote} % Required for margin notes
\usepackage{wallpaper} % Required to set each page to have a background
\usepackage{lastpage} % Required to print the total number of pages
\usepackage[left=1.3cm,right=4.6cm,top=1.8cm,bottom=4.0cm,marginparwidth=3.4cm]{geometry} % Adjust page margins
\usepackage{amsmath} % Required for equation customization
\usepackage{amssymb} % Required to include mathematical symbols
\usepackage{xcolor} % Required to specify colors by name
\usepackage{listings}

\usepackage{fancyhdr} % Required to customize headers
\setlength{\headheight}{80pt} % Increase the size of the header to accommodate meta-information
\pagestyle{fancy}\fancyhf{} % Use the custom header specified below
\renewcommand{\headrulewidth}{0pt} % Remove the default horizontal rule under the header

\setlength{\parindent}{0cm} % Remove paragraph indentation
\newcommand{\tab}{\hspace*{2em}} % Defines a new command for some horizontal space

\newcommand\BackgroundStructure{ % Command to specify the background of each page
\setlength{\unitlength}{1mm} % Set the unit length to millimeters

\definecolor{amaranth}{rgb}{0.9, 0.17, 0.31}
\definecolor{babyblueeyes}{rgb}{0.63, 0.79, 0.95}
\definecolor{beige}{rgb}{0.96, 0.96, 0.86}
\definecolor{bittersweet}{rgb}{1.0, 0.44, 0.37}
\definecolor{black}{rgb}{0.0, 0.0, 0.0}
\definecolor{bleudefrance}{rgb}{0.19, 0.55, 0.91}
\definecolor{bostonuniversityred}{rgb}{0.8, 0.0, 0.0}
\definecolor{brightube}{rgb}{0.82, 0.62, 0.91}
\definecolor{darkseagreen}{rgb}{0.56, 0.74, 0.56}
\definecolor{lavender}{rgb}{0.9, 0.9, 0.98}
\definecolor{mayablue}{rgb}{0.45, 0.76, 0.98}
\definecolor{cadmiumgreen}{rgb}{0.0, 0.42, 0.24}
\definecolor{almond}{rgb}{0.94, 0.87, 0.8}
\definecolor{antiquewhite}{rgb}{0.98, 0.92, 0.84}
\definecolor{ashgrey}{rgb}{0.7, 0.75, 0.71}
\definecolor{babyblueeyes}{rgb}{0.63, 0.79, 0.95}
\definecolor{beige}{rgb}{0.96, 0.96, 0.86}
\definecolor{blond}{rgb}{0.98, 0.94, 0.75}
\definecolor{cream}{rgb}{1.0, 0.99, 0.82}
\definecolor{eggshell}{rgb}{0.94, 0.92, 0.84}

\setlength\fboxsep{0mm} % Adjusts the distance between the frameboxes and the borderlines
\setlength\fboxrule{0.5mm} % Increase the thickness of the border line
\put(10, 10){\fcolorbox{black}{darkseagreen!10}{\framebox(155,247){}}} % Main content box
\put(165, 10){\fcolorbox{black}{blond!10}{\framebox(37,247){}}} % Margin box
\put(10, 262){\fcolorbox{black}{amaranth!10}{\framebox(192, 25){}}} % Header box
\put(170, 263){\includegraphics[height=23mm,keepaspectratio]{csm}} % Logo box - maximum height/width: 
}

%----------------------------------------------------------------------------------------
%	HEADER INFORMATION
%----------------------------------------------------------------------------------------

\fancyhead[L]{
\begin{tabular}{l l | l l} % The header is a table with 4 columns
\textbf{CSC447:} & Parallel Programming &  \textbf{Name:} & Samer Saber  \\ % Project name and page count
\textbf{Lab 2:}  & Data Race &   \textbf{ID:}  & 123456789 and 987654321\\ % Project name and page count
\textbf{Date:}&   \today &  \textbf{Page:} & \thepage/\pageref{LastPage}  \\ % Project name and page count
\textbf{Spring 2021} & & & \\ % Version and reviewed date
\end{tabular}}


 

%&  & \textbf{Date} & 27/11/2012 \\ % Job number and last updated date


%----------------------------------------------------------------------------------------

\begin{document}

\AddToShipoutPicture{\BackgroundStructure} % Set the background of each page to that specified above in the header information section

%----------------------------------------------------------------------------------------
%	DOCUMENT CONTENT
%----------------------------------------------------------------------------------------

\subsection*{Abstract}
Give a brief summary of the problem, experimental procedure, and what was done.

\section{Introduction}

Introduce the lab and what what was required to do.  You may use overleaf.com to create your reports.  A much nicer way of doing things.

\section{Implementation}

Explain what was done.  Please explain what was done.   Check the grading rubric to ensure that all points were tackled.  This is how you could include your code:

\begin{lstlisting}[language=Python, caption=Python Example]
import numpy as np
    
def incmatrix(genl1,genl2):
    m = len(genl1)
    n = len(genl2)
    M = None #to become the incidence matrix
    VT = np.zeros((n*m,1), int)  #dummy variable
    
    #compute the bitwise xor matrix
    M1 = bitxormatrix(genl1)
    M2 = np.triu(bitxormatrix(genl2),1) 

    for i in range(m-1):
        for j in range(i+1, m):
            [r,c] = np.where(M2 == M1[i,j])
            for k in range(len(r)):
                VT[(i)*n + r[k]] = 1;
                VT[(i)*n + c[k]] = 1;
                VT[(j)*n + r[k]] = 1;
                VT[(j)*n + c[k]] = 1;
                
                if M is None:
                    M = np.copy(VT)
                else:
                    M = np.concatenate((M, VT), 1)
                
                VT = np.zeros((n*m,1), int)
    
    return M
\end{lstlisting}

\section{Experimental Platform}
Explain the experimental setup including parameters, hardware used, and compiler

\section{Results}
Presents the results that were obtained.

\section{Discussion}

As the semester progresses, you would be expected to further analyze your data critically and drawing valid inferences from data, a vital skill for scientists.  include the following whenever possible:
\begin{enumerate}

	\item Numerical Analysis
	\item Graphical Analysis using Excel to construct graphs or plots.  Provide critical analysis, explaining if your graphical analysis agree with your calculations.
\end{enumerate}


\subsubsection{Example Table}
This is a sample table.

\begin{table}[h]
\begin{center}
\begin{tabular}{|l||l|l|l|}
\hline
Member & Designation & Category & $\lambda_e$\\
\hline
Active Link & 360UB57 & 1 & 25 \\
Collector Beam & 360UB57 & 2 & 30 \\
Column & 310UC137 & 2 & 30 \\
Brace & 250UC73 & 3A & 40 \\
\hline
\end{tabular}
\end{center}
\caption{This is a caption}
\end{table}

And this is a sample equation:

\begin{equation*}
\tag{Eq. 1}
e \leqslant 1.6M_s/V_v
\end{equation*}



\section{Conclusion}
And so on\dots

%----------------------------------------------------------------------------------------

\end{document}